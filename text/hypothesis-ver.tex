\chapter{Weryfikacja problemów badawczych}
\label{sec:hyp-ver}
\section{Zależność między satysfakcją, a zaangażowaniem}
\subsection{Badanie}
Do badania zależności między satysfakcją, a zaangażowaniem wykorzystano współczynnik korelacji (przy istotności statystycznej 0,05). Obliczenia dla każdej z par wymiarów, jak i sumarycznych wyników znajdują się w Tabeli \ref{tab:jss-uwes-correl}.

\begin{table}[h!]
\begin{center}
\begin{tabular}{l || c c c | c}
  & Wigor & Oddanie & Absorpcja & Zaangażowanie \\ \hline \hline
Płaca & 0,18 & 0,32 & 0,24 & 0,27 \\
Awanse & 0,23 & 0,34 & 0,16 & 0,26 \\
Nadzór & 0,17 & 0,37 & 0,21 & 0,27 \\
Świadczenia pracownicze & 0,19 & 0,19 & 0,08 & 0,17 \\
Wyrażanie uznania & 0,27 & 0,3 & 0,24 & 0,29 \\
\textcolor{Mahogany}{Organizacja pracy} & \textcolor{Mahogany}{-0,04} & \textcolor{Mahogany}{-0,1} & \textcolor{Mahogany}{-0,03} & \textcolor{Mahogany}{-0,06} \\
Współpracownicy & 0,32 & 0,39 & 0,22 & 0,34 \\
\textcolor{OliveGreen}{Charakter pracy} & \textcolor{OliveGreen}{0,62} & \textcolor{OliveGreen}{0,79} & \textcolor{OliveGreen}{0,53} & \textcolor{OliveGreen}{0,71} \\
Komunikacja & 0,31& 0,38 & 0,33 & 0,38 \\ \hline
Satysfakcja & 0,36 & 0,48 & 0,32 & 0,42 \\ \hline
\end{tabular}
\end{center}
\caption{Korelacja między zaangażowaniem w pracę, a satysfakcję z pracy wraz z ich wymiarami.}
\label{tab:jss-uwes-correl}
\end{table}

\subsection{Dyskusja}
Jak widać w Tabeli \ref{tab:jss-uwes-correl}, między satysfakcją z pracy, a zaangażowaniem w pracę jest tylko średniej mocy zależność (wartość 0,42). Częściowo potwierdza to hipotezę postawioną w rozdziale \ref{sec:hypothesis-relation} ponieważ nie jesteśmy w stanie odrzucić istnienia zależności, ale też z całą pewnością jej potwierdzić. Wydawać by się mogło, że zgodnie z teorią satysfakcji opisaną w rozdziale \ref{sec:theory-sat}:

\begin{iquote}
  Zależy ona [\textit{satysfakcja z pracy}] od wielu czynników związanych z pracą (\ldots) do poczucia samospełnienia przy realizacji codziennych zadań. \cite[str. 296]{SchultzSat}
\end{iquote}

istnieje powiązanie z zaangażowaniem w pracę. W końcu fragment ten wydaje się być zbieżny z teorią Kahna odnoszącą się do wyrażania siebie w pracy (patrz rozdział \ref{sec:theory-eng-kahn}). W szczególności przytoczony cytat przypomina o wymiarach \textit{absorpcji} i \textit{wigoru}. 

Co ciekawe nie ma kompletnie relacji między wymiarem satysfakcji -- \textit{organizacja pracy}, a \textit{zaangażowaniem}. Podobna sytuacja jest ze \textit{współpracownikami} czy \textit{nadzorem}. Jest to niezgodne z teorią opisaną w rozdziałach o wpływie zaangażowania na pracownika i vice versa (rozdział \ref{sec:theory-eng-infl} -- ,,Zasoby w pracy'', rozdział \ref{sec:thoery-eng-infl2} -- ,,Pozytywne zachowania''). Z Tabeli
\ref{tab:jss-uwes-correl} wynika, że organizacja pracy kompletnie nie ma wpływu na poziom naszej energii, stopień skupienia na zadaniu czy na oddaniu
wykonywanej pracy. Pomimo tego, że to właśnie odpowiednie warunki i środowisko pracy powinno sprzyjać choćby wymiarowi \textit{absorpcja}. Jednak zależność między \textit{absorpcją}, a \textit{zaangażowaniem} wynosi
-0,03, zdecydowany brak jakiejkolwiek relacji przyczynowo-skutkowej. Wszystkie odpowiedzi badanych bazują na ich percepcji i ocenie różnych aspektów pracy. Respondenci mogą być nieświadomi braków, na którymkolwiek z wymiarów, które później mogą wpływać na inne aspekty czy emocje związane z pracą. W związku z tym warto zadać pytanie: czy niska wartość korelacji między \textit{organizacja pracy}, a \textit{zaangażowaniem} wskazuje, że badani nie są świadomi jak ten aspekt pracy wpływa na ich \textit{absorpcję} podczas pracy? Idąc dalej, jakie mają oczekiwania wobec
\textit{organizacji pracy}?

Z drugiej strony okazało się, że istnieje zależność między aspektem satysfakcji --\textit{charakter pracy}, a \textit{zaangażowaniem} (wartość 0,71). Korelacje dla wspomnianego wymiaru satysfakcji oraz każdego z poszczególnych wymiarów zaangażowania są także wysokie:
\begin{itemize}
  \item r(\textit{charakter pracy}, \textit{oddanie}) = 0,79
  \item r(\textit{charakter pracy}, \textit{wigor}) = 0,62
  \item r(\textit{charakter pracy}, \textit{absorpcja}) = 0,53
\end{itemize}
Jak widać rodzaj wykonywanych zadań ma największy wpływ na oddanie pracy, czyli na uczucie dumy z charakteru pracy oraz jej sensowności w oczach pracownika. Oznacza to, że jeżeli ludzie lubią i cenią to co robią, sprzyja to ich oddaniu. 

Następny w kolejności wynik ma \textit{wigor}. Wydaje się być naturalnym, że skoro jesteśmy zadowoleni z charakteru pracy łatwiej nam włożyć w nią więcej energii, mamy więcej motywacji i siły do pokonywania przeszkód, które się pojawią w trakcie
realizacji zadań (i vice versa). 

Ostatnim wymiarem, ze średniej mocy zależnością, jest \textit{absorpcja}. Skoro lubimy wykonywać naszą pracą, nie wydaje się dziwnym to, że czas podczas jej wykonywania czas płynie bardzo szybko i potrafimy się całkowicie na niej skupić. 

Należy jednak postawić pytanie: dlaczego \textit{absorpcja} ma tylko średniej mocy relację z \textit{charakterem pracy}? Dlaczego kolejność siły zależności wygląda w ten sposób? Prawdopodobnie ma to związek z formą postawionych pytań w kwestionariuszu \emph{JSS}. Połowa pytań dla wymiaru \textit{charakter pracy} to:
\begin{iquote}
  Czasami myślę, że moja praca jest bez sensu.
\end{iquote}
\begin{iquote}
  Jestem dumny z mojej pracy. (patrz dodatek \ref{sec:jss-text})
\end{iquote}
Wydają się one odzwierciedlać idealnie \textit{oddanie}, czyli wymiar z najsilniejszą korelacją. Kolejne dwa pytania odnoszą się do lubienia pracy i odczuwania przyjemności podczas wykonywania zadań, które faktycznie mogą mieć związek z \textit{wigorem} i \textit{absorpcją}.


\section{Satysfakcja z pracy wśród pracowników sektora IT, a polskie normy}
\subsection{Badanie}
\begin{hyp}
  Pracownicy sektora IT są bardziej zadowoleni z pracy niż przeciętni Polacy.
  \label{hip:sat}
\end{hyp}

W celu zweryfikowania Hipotezy \ref{hip:sat} przeprowadzono test różnic dla dwóch zmiennych niezależnych przy pomocy statystyki Z:

\begin{equation}
  Z = \frac{\overline{X_2} - \overline{X_1}}{\sqrt{\frac{\sigma^2_2}{N_2}+\frac{\sigma^2_1}{N_1}}}
\end{equation}

Dla badań populacji estymatory ($\overline{X}$) to średnie ($\mu$) dla badanych cech. Czyli badaną hipotezę statystyczną można przedstawić w następujący sposób:

\begin{equation}
  H0: \mu_{PL} = \mu_{IT} \qquad H1: \mu_{PL} < \mu_{IT}
\end{equation}

\begin{table}[h!b]
  \begin{center}
    \begin{tabular}{l | c c c }
      Satysfakcja & N & Śr. & War. \\ \hline
      sektor IT & 73 & 4,16 & 0,50 \\
      Polacy & 521 & 3,70 & 0,38 \\
    \end{tabular}
  \end{center}
  \caption{Statystyki opisowe potrzebne do wyliczenia statystyki Z.}
  \label{tab:jss-norms-data}
\end{table}

Do obliczenia statystyki Z wykorzystano dane z Tabeli \ref{tab:jss-norms-data}. Przy czym wybrano próg istotności statystycznej $\alpha = 0,05$, czyli $Z_{kryt.} = -1,64$.

\begin{equation}
  Z = \frac{\overline{X_{PL}} - \overline{X_{IT}}}{\sqrt{\frac{\sigma^2_{PL}}{N_{PL}}+\frac{\sigma^2_{IT}}{N_{IT}}}} = \frac{\mu_{PL} - \mu_{IT}}{\sqrt{\frac{\sigma^2_{PL}}{N_{PL}}+\frac{\sigma^2_{IT}}{N_{IT}}}} = \frac{3,70 - 4,16}{\sqrt{\frac{0,38}{521}+\frac{0,50}{73}}} = -5,29 
\end{equation}

Na podstawie niespełniania zależności:

\begin{equation}
  Z_{kryt.} < Z
\end{equation}

można odrzucić $H0: \mu_{PL} = \mu_{IT}$ oraz przyjąć $H1: \mu_{PL} < \mu_{IT}$ przy istotności statystycznej na poziomie $\alpha = 0,05$.

\subsection{Dyskusja}
Jak widać z powyższych badań pracownicy sektora IT są bardziej zadowoleni z pracy niż przeciętni Polacy. 

Nasza grupa badanych jest dosyć młoda (patrz rozdział \ref{sec:group-age}), czyli teoretycznie ich zadowolenie powinno być niższe (zgodnie z rozdziałem \ref{sec:theory-sat} -- ,,Wpływ cech indywidualnych pracownika``). Warto tutaj przypomnieć fragment definicji satysfakcji z pracy z rozdziału \ref{sec:theory-sat}:
\begin{iquote}
  Nasza motywacja i aspiracje, oraz sposób ich zaspokajania przez pracę, także wpływają na postawy wobec pracy. [w tym satysfakcję z pracy] \cite[str. 296]{SchultzSat}
\end{iquote}
Widzimy, że sektor IT ma bardzo dużo do zaoferowania osobom z krótkim doświadczeniem zawodowym (patrz rozdział \ref{sec:group-exp}). Nawet w porównaniu z ogółem Polaków (przekrój lat doświadczenia o wiele szerszy), wypadają lepiej jeżeli chodzi o poziom satysfakcji z pracy.

Wynik ten jest także ciekawy kiedy weźmiemy pod uwagę teorię wpływu Locke'a (rozdział \ref{sec:theory-sat-locke}) o dopasowaniu oczekiwań, a zastaną sytuacją w pracy, jako główny czynnik wpływający na satysfakcję z pracy. Studia informatyczne są bardzo praktycznymi studiami. Dzięki przystępnym cenom sprzętu komputerowego oraz udostępnianiu dużej części narzędzi za darmo w Internecie (ruch
\href{http://en.wikipedia.org/wiki/Open-source_software}{OpenSource} oraz \href{http://en.wikipedia.org/wiki/Free_software}{FreeSoftware}) w bardzo łatwy sposób można odtworzyć środowisko pracy w laboratorium na uczelni, jak i w domu. Dzięki projektom zaliczeniowym w grupach studenci uczą się pracować w zespołach, jak to ma miejsce w środowisku pracy. Możliwe, że dzięki takiemu procesowi nauczania, który jest zbieżny z rzeczywistością zastaną w pracy, oczekiwania ludzi
wkraczających na rynek pracy nie odbiegają zbytnio od zastanych. Stąd może wynikać większa satysfakcja z pracy wśród pracowników sektora IT niż normy dla wszystkich Polaków. W końcu normy tworzony są na podstawie uśredniania po wszystkich grupach zawodowych (grupa ogółu Polaków).

Natomiast patrząc na obecne wyniki z punktu widzenia teorii Herzberga (rozdział \ref{sec:theory-sat-herz}) wszystkie czynniki higieny jakie badaliśmy są spełnione (większość respondentów jest usatysfakcjonowana z odpowiadających wymiarów pracy). Ponadto z badanych motywatorów, z jednego są usatysfakcjonowani (docenienie, patrz wymiar \textit{wyrażanie uznania} w teście \emph{JSS}), z jednego mają mieszane uczucia (możliwości promocji, patrz wymiar \textit{awans} w teście \emph{JSS}). Z czego można wywnioskować, że
grupa respondentów powinna być raczej zadowolona ze swojej pracy, skoro brak jest czynników powodujących zmniejszenie satysfakcji (niezadowolenie na czynnikach higieny) oraz istnieją czynniki powodujące ich wzrost (zadowolenie na chociaż jednym motywatorze). 

Poniżej znajduje się lista motywatorów i czynników higieny badanych przez \emph{JSS} oraz wyniki jakie zostały uzyskane podczas badania.

\begin{itemize}
  \item motywator \textit{docenienie} $\rightarrow$ wymiar \textit{wyrażanie uznania} (śr. 15,86),
  \item motywator \textit{możliwości promocji} $\rightarrow$ wymiar \textit{awanse} (śr. 13,47),
  \item czynnik higieny \textit{płaca} $\rightarrow$ wymiar \textit{płaca} (śr. 15,99),
  \item czynnik higieny \textit{nadzór} $\rightarrow$ wymiar \textit{nadzór} (śr. 19.18),
  \item czynnik higieny \textit{organizacja w firmie} $\rightarrow$ wymiar \textit{organizacja pracy} (śr. 15,51).
\end{itemize}

Co ciekawe, pomimo tego że mamy młodą grupę badanych, są oni usatysfakcjonowani swoją pracę. Jest to sprzeczne z opisem wypływu wieku na zadowolenie z pracy z rozdziału \ref{sec:theory-sat-age}. Jednak jeżeli zwrócimy uwagę na wymiar \textit{awansów}, gdzie tyle samo osób jest zadowolonych i tyle samo niezadowolonych, oraz spojrzymy na doświadczenie naszej grupy (co najwyżej kilkuletnie) i zajmowane stanowiska (głównie specjaliści różnego stopnia), widzimy, że pomimo młodego wieku
osoby te mają szanse na awans oraz pracują na znaczących pozycjach (większość to specjaliści).

Za to zgodnie z opisem zależności dla zdolności poznawczych w rozdziale \ref{sec:theory-sat-age}, osoby badane są głównie zadowolone ze swojej pracy. Należy tutaj pamiętać, że większość osób pracuje jako specjaliście różnego szczebla, więc ich zdolności poznawcze powinny być wysokie.

Potwierdzają się także badania odnośnie zachowań prospołecznych (patrz rozdział \ref{sec:theory-sat-infl} -- ,,Zachowania prospołeczne i nieproduktywne``). Większość osób jest zadowolonych ze swojej pracy i jednocześnie bardzo wysoko ocenia swoich współpracowników (wymiar \textit{współpracownicy}).
\section{Zaangażowanie w pracę wśród pracowników sektora IT, a polskie normy}
\subsection{Badanie}
\begin{hyp}
  Pracownicy sektora IT są bardziej zaangażowani w pracę niż przeciętni Polacy.
  \label{hip:eng}
\end{hyp}

Podobnie jak w przypadku satysfakcji, aby zbadać prawdziwość Hipotezy \ref{hip:eng} przeprowadzono test różnic dla dwóch zmiennych niezależnych przy pomocy statystyki Z:

\begin{equation}
  Z = \frac{\overline{X_2} - \overline{X_1}}{\sqrt{\frac{\sigma^2_2}{N_2}+\frac{\sigma^2_1}{N_1}}}
\end{equation}

Badana hipoteza statystyczna  to:

\begin{equation}
  H0: \mu_{PL} = \mu_{IT} \qquad H1: \mu_{PL} < \mu_{IT}
\end{equation}

\begin{table}[h!]
  \begin{center}
    \begin{tabular}{l | c c c }
      Zaangażowanie & N & Śr. & War. \\ \hline
      sektor IT & 73 & 3,76 & 1,04 \\
      Polacy & 1438 & 3,86 & 0,90 \\
    \end{tabular}
  \end{center}
  \caption{Statystyki opisowe potrzebne do wyliczenia statystyki Z.}
  \label{tab:uwes-norms-data}
\end{table}

Do obliczenia statystyki Z wykorzystano dane z Tabeli \ref{tab:uwes-norms-data} przy progu istotności statystycznej $\alpha = 0,05$, czyli $Z_{kryt.} = -1,64$.

\begin{equation}
  Z = \frac{\overline{X_{PL}} - \overline{X_{IT}}}{\sqrt{\frac{\sigma^2_{PL}}{N_{PL}}+\frac{\sigma^2_{IT}}{N_{IT}}}} = \frac{\mu_{PL} - \mu_{IT}}{\sqrt{\frac{\sigma^2_{PL}}{N_{PL}}+\frac{\sigma^2_{IT}}{N_{IT}}}} = \frac{3,86- 3,76}{\sqrt{\frac{0,90}{1438}+\frac{1,04}{73}}} = 0,19
\end{equation}

Na podstawie spełnienia zależności:

\begin{equation}
  Z_{kryt.} < Z
\end{equation}

można przyjąć $H0: \mu_{PL} = \mu_{IT}$ oraz odrzucić $H1: \mu_{PL} < \mu_{IT}$ przy istotności statystycznej na poziomie $\alpha = 0,05$.

\subsection{Dyskusja}
Jak widać z powyższego testu statystycznego, pracownicy sektora IT są tak samo zaangażowani w pracę jak przeciętni Polacy.

Wyrażając wynik przy pomocy definicji Kahna (patrz rozdział \ref{sec:theory-eng-kahn}), informatycy nie wyrażają siebie w pracy lepiej niż przeciętni Polacy. Pytanie, czy istnieje powód, aby było na odwrót? W obecnych czasach sami decydujemy o wyborze zawodu, a co za tym idzie jak będziemy wyrażać się pracy. Jak widać, statystycznie rzecz biorąc, pracownicy sektora IT nie różnią się tutaj od przeciętnego Polaka. Wynika z tego, że podobny odsetek osób jest dopasowanych do
wykonywanego zawodu (tzw. odpowiedniość pracy).

Z kolei z perspektywy teorii Maslacha, Jackson i Leitera (patrz rozdział \ref{sec:model-maslach}) badana grupa nie jest bardziej wypalona w pracy niż ogół. Mają podobną energię w pracy, stosunek do niej (cynizm lub entuzjazm) oraz wykazują taką samą skuteczność podczas wykonywania zadań. Czyli badana grupa nie posiada żadnych specyficznych cech, które wyróżniałyby ich na tle ogółu Polaków i predysponowały do zwiększonego zaangażowania w pracę. Co ciekawe grupa badanych
jest młoda oraz większość osób pracuje na stanowiskach specjalistycznych (pomimo wieku). Jednak nie wpływa to w żaden sposób na
ilość osób zaangażowanych czy wypalonych osób. Wydawać by się mogło, że młode osoby powinny później ulegać wypaleniu zawodowemu. Jednak nie ma to odzwierciedlenia w niniejszych badaniach.

Patrząc na ostatnią teorię Schaufelliego oraz Bakkera (patrz rozdział \ref{sec:model-schauffeli}) możemy powiedzieć, że respondenci z podobną częstotliwością oraz w podobny sposób odczuwają pozytywne uczucia spełnienia związane z wykonywaną pracą, a w szczególności \textit{wigor}, \textit{oddanie} i \textit{absorpcję}. Co ciekawe nie ma to nic wspólnego z odpowiedniością pracy (patrz wysokie wyniki z satysfakcji na wymiarze \textit{charakter pracy}). Tym bardziej
jeżeli chodzi o ostatni wymieniony wymiar zaangażowania związany z wgłębieniem się w zadania oraz skupieniu nad pracą. Wydaje się to być zaskakujące.
