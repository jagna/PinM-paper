\section{Metoda przeprowadzenia badań}
Badanie zostało przeprowadzone jako badanie internetowe. Trwało dokładnie jeden miesiąc od 29 IV do 29 V 2011. Do przygotowania ankiety zostały użyte polskie tłumaczenia \emph{Job Satisfaction Survey} oraz \emph{Ultrecht Work Engagement Survey} przygotowane przez TODO-PROMOTOR (w załączniku pełna treść pytań TODO-REF). Pytania zostały wprowadzone do instancji darmowego oprogramowania do przeprowadzania ankiet w Internecie -- LimeSurvey (TODO-URL). Ankieta jest nadal dostępna pod
adresem survey.alewandowska.pl (TODO-URL). 

TODO-IMG!!! pierwsza strona + strona z wynikami

Odnośnik do gotowej ankiety internetowej został rozesłany drogą emailową na listy emailowe kilku instytucji i organizacji, w tym:
\begin{itemize}
\item lista absolwentów kierunku Informatyka na Wydziale Informatyki i Zarządzania Politechniki Poznańskiej -- rok ukończenia 2008,
\item lista pracowników Poznańskiego Centrum Superkomputerowo-Sieciowego,
\item lista pracowników Allegro.
\end{itemize}
Oczywiście wszyscy byli zachęcani do rozsyłania ankiety dalej, więc osoby biorące udział w badaniu nie muszą, ale mogą być ograniczone tylko do wskazanych firm. Wypełnienie ankiety było dobrowolne, przy czym wszystkie pytania w ankiecie były obowiązkowe, także dotyczące danych demograficznych.
