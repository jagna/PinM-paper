\chapter{Zaangażowanie w pracę}
Przed zdefiniowaniem zaangażowania w pracę z punktu widzenia psychologii, warto przytoczyć potoczną definicję tego słowa. Wg Słownika Języka Polskiego PWN
\begin{iquote}
zaangażować się -- włożyć w coś wiele wysiłku, czasu lub pieniędzy \cite{web:pwn-eng}
\end{iquote}
Zawężając powyższą definicję do pracy
\begin{iquote}
zaangażować się w pracę -- włożyć w pracę wiele wysiłku, czasu lub pieniędzy
\end{iquote}

Zaangażowanie w pracę nie jest ciągłym stanem podczas wykonywania pracy. Jego poziom może się zmieniać zależnie od tygodnia, czy dnia pracy \cite{bakker2010weekly}. Natomiast jest to powtarzający się stan, co jest wiązane z charakterem pracy i rodzajem organizacji, w której pracownik pracuje \cite{macey2008meaning}.

W przeciwieństwie do wypalenia zawodowego, zmęczenie w zaangażowaniu w pracę jest pozytywnym odczuciem powiązanym z dobrym wykonaniem zadania.

Należy także podkreślić, że zaangażowani pracownicy nie są pracoholikami. Pracoholicy pracują intensywnie ze względu na wewnętrzny impuls, który jest nie do odparcia. Natomiast osoby zaangażowane w pracę, lubią to co robią i to stanowi ich motywację do intensywnej pracy. Poza tym posiadają życie osobiste poza pracą i potrafią się cieszyć aktywnościami niezwiązanymi z pracą.

Ze względu na to, że zaangażowanie w pracę jest dość młodym, a co za tym idzie rzadko póki co badanym konceptem (w przeciwieństwie do np.: satysfakcji z pracy), obecnie jest mało badań wiążących zaangażowanie w prace z cechami osobowości pracowników.
\section{Modele teoretyczne}
Poniżej zaprezentowano wybrane teorie związane z zaangażowaniem w pracę.

\subsection{Kahn (1990)}
\label{sec:theory-eng-kahn}
Pierwszy model stworzony przez W. Kahna w 1990  roku \cite{kahn1990psychological}, wychodzi z założenia, że zaangażowanie to konstrukt związany z wyrażaniem siebie fizycznie, emocjonalnie i poznawczo podczas wykonywania pracy. W 2004 na podstawie tego modelu May, Gilson i Harter zoperacjonalizowali ten konstrukt i opracowali test psychologiczny badający zaangażowanie w pracę \cite{may2004psychological}.

\subsection{Maslach, Jackson i Leiter (1996)}
\label{sec:model-maslach}
Teoria ta definiuje zaangażowanie jako przeciwieństwo wypalenia zawodowego i tworzy kontinuum między tymi dwoma pojęciami \cite{maslach1997truth}. Osoby wypalone zazwyczaj są wyczerpane i cynicznie podchodzą do pracy, natomiast zaangażowane są pełne energii i entuzjastycznie podchodzą do wykonywanych zadań. Dodatkowo istnieje różnica między wydajnością uzyskaną przy tak samo włożonym wysiłku w pracę. W efekcie otrzymujemy 3 wymiary:
\begin{itemize}
\item wyczerpanie $\leftrightarrow$ energia,
\item cynizm $\leftrightarrow$ entuzjazm,
\item mała skuteczność $\leftrightarrow$ duża skuteczność,
\end{itemize}
których pozytywny koniec odpowiada zaangażowaniu, a negatywny -- wypaleniu zawodowemu. Oczywiście osoby bardziej wypalone osiągają słabsze rezultaty.

Teoria ta spotyka się z głosami krytyki. Istnieją wątpliwości czy faktycznie zaangażowanie można tak jednoznacznie przedstawić jako przeciwieństwo wypalenia zawodowego. Czy na pewno kiedy pracownik nie jest zaangażowany oznacza to wypalenia zawodowe? Intuicyjnie chce się odpowiedzieć, że nie każda praca musi być fascynująca. Aby móc zbadać faktyczną zależność między zaangażowaniem, a wypaleniem należało stworzyć nowy konstrukt nie opierający się na teorii wypalenia zawodowego.

\subsection{Schauffeli i Bakker (2001)}
\label{sec:model-schauffeli}
Model ten odrywa się od definiowana zaangażowania jako przeciwieństwa wypalenia zawodowego \cite{schaufeli2002measurement}. Jakkolwiek nie odrzuca możliwego powiązania między tymi dwoma konstruktami. 

Teoria ta definiuje zaangażowanie jako pozytywne uczucie spełnienia związane z wykonywaną pracą, które można opisać przez trzy elementy:
\begin{itemize}
\item wigor -- osoba jest energetyczna, odporna psychicznie na problemy podczas wykonywania zadań i wkłada dużo wysiłku w pracę,
\item oddanie -- pracownikowi zależy na pracy, uważa ją za znaczącą, inspirującą, pełną wyzwań oraz jest dumny z tego co robi,
\item absorpcję -- osoba jest w pełni skupiona i wciągnięta w wykonywana zadania; nie czuje upływu czasu i ma problemy z oderwaniem się od zadań.
\end{itemize}
Przy czym wigor i oddanie można powiązać z modelem wypalenia jako przeciwległe stany wyczerpania i cynizmu. Natomiast właśnie absorpcja jest wymiarem, który stanowi o różnicy między wypaleniem, a zaangażowaniem. W absorpcji nie chodzi o wydajność i skuteczność w pracy, a o przyjemne uczucie zagłębienia się w wykonywane zadania.

\section{Sposoby pomiaru}
Na podstawie teorii Maslacha, Jackson i Leitera (zaangażowanie jako stan przeciwny do wypalenia zawodowego, rozdział \ref{sec:model-maslach}) wykorzystuje się kwestionariusz do badania wypalenia zawodowego -- Maslach Burnout Inventory (\emph{MBI}). Jeżeli osoba uzyska tam ,,złe" wyniki, tzn. mało punktów na skali wypalenia i cynizmu, a dużo na skali wydajności w pracy, wówczas uznaje się, że pracownik jest zaangażowany w pracę.

Natomiast na podstawie teorii Schauffeli'ego i Bakkera (rozdział \ref{sec:model-schauffeli}) stworzono \emph{Utrecht Work Engagement Scale} (\emph{UWES}), który bada 3 elementy zaangażowania: wigor, absorpcję i oddanie. Rzetelność i ważność testu została potwierdzona przez kilka badań.

\section{Wpływ cech pracy i cech indywidualnych pracownika}
\label{sec:theory-eng-infl}
\paragraph{Zasoby w pracy}
Istnieje pozytywna korelacja między dostępnymi zasobami w pracy, a zaangażowaniem pracownika. Zasoby w miejscu pracy, takie jak: różnorodność zadań, okresowa ocena, społeczne wsparcie od współpracowników/przełożonego, wpływają pozytywnie na zaangażowanie w pracę, a co za tym idzie, na dobre wykonywanie zadań w pracy. Potwierdzają to badania na grupie fińskich dentystów \cite{hakanen2008positive}.
\paragraph{Zasoby osobiste}
Zasoby, które wpływają na wyższy poziom opanowania i kontroli w pracy, takie jak: optymizm, odporność na przeszkody/porażki oraz pewność własnych umiejętności, korelują pozytywnie z zaangażowaniem w pracę. Potwierdzają to badania wykonane na grupie holenderskich techników \cite{xanthopoulou2007role}.

Dodatkowo wg Langelaana, Bakkera, Van Doornena i Schaufeliego osoby zaangażowane wykazują się niskim poziomem neurotyczności, a wysokim -- ekstrawersji. Czyli osoby takie są zbalansowane emocjonalnie, bardziej odporne na stres, otwarte na kontakty towarzyskie w pracy oraz pozytywnie do niej podchodzą \cite{langelaan2006burnout}. 

\section{Wpływ zaangażowania na pracowników i ich zachowania}
\label{sec:thoery-eng-infl2}
\paragraph{Wypalenie zawodowe}
Wg González-Roma'y, Schaufeliego, Bakkera, Lloreta zaangażowani w pracę pracownicy wykazują niski poziom wypalenia zawodowego \cite{gonzlez2006burnout}. Biorąc pod uwagę teorie związane z zaangażowaniem nie jest to zaskakujący wniosek.
\paragraph{Zdrowie}
Co ciekawe, wg badań z 2008, osoby zaangażowane w pracę są zdrowsze psychicznie i fizycznie, co jest zgodne z koncepcją całościowego podejścia do zdrowia człowieka (stan fizyczny i psychiczny wpływają na siebie nawzajem) \cite{schaufeli2008workaholism}.
\paragraph{Wydajność}
Ponadto kilka badań \cite{xanthopoulou2008working,xanthopoulou2009work} wykazało, że zaangażowani pracownicy są bardziej wydajni w pracy. Wydaje się to zgodne z teorią i innymi badaniami. Osoby te często doświadczają pozytywnych odczuć podczas wykonywania zadań w pracy, są zdrowsze fizycznie i psychicznie, co przekłada się na kontakty ze współpracownikami, czy też klientami.
\paragraph{Pracoholizm}
Ponadto badania Schaufeliego, Tarisa i Van Rhenena wykazują, że pracoholizm nie jest w ogóle powiązany z zaangażowaniem w pracę \cite{schaufeli2008workaholism}. Z zewnątrz, oba te aspekty wyglądają podobnie: pracownicy pracują ciężko i z oddaniem dla swojej firmy, jednak związane są z innymi skutkami ubocznymi. Pracoholicy nie mają życia poza pracą, co pokutuje na ich zdrowiu fizycznym i psychicznym. Natomiast osoby zaangażowane posiadają wsparcie w innych dziedzinach życia nie związanych z pracą (rodzina,
znajomi, zainteresowania, itp.) oraz czują się lepiej fizycznie i psychicznie.
\paragraph{Odpoczynek}
Sonnetag w 2003 wykazał jak odpoczynek jest istotny dla zaangażowania \cite{sonnentag2003recovery}. Okazało się, że stopień zaangażowania w zadania danego dnia jest uzależniony od tego, jak dnia poprzedniego pracownik odpoczął po pracy/zdołał się zregenerować.
\paragraph{Zdrowe życie rodzinne}
W 2003 wykazano, że pozytywne odczucia z życia rodzinnego są przenoszone do pracy i wpływają pozytywnie na zaangażowanie \cite{montgomery2003work}. Stosunek ten jest obustronny, tzn. negatywne odczucia z pracy wpływają źle na życie rodzinne. 

Co więcej wykazano w tym samym roku, że stopień zaangażowania przechodzi w obie strony między małżonkami \cite{bakker2003crossover}. Zaangażowane w pracę żony wpływa na zaangażowanie w pracę męża i vice versa.
\paragraph{Pozytywne zachowania}
Istnieją badania wskazujące na korzystny wpływ zaangażowania na pozytywne zachowania w pracy oraz postawy wobec organizacji:
\begin{itemize}
  \item satysfakcja z pracy, przywiązanie organizacyjne, niskie intencje odejścia \cite{demerouti2001job,salanova2000burnout,schaufeli2008workaholism},
  \item własna inicjatywa, motywacja do nauki \cite{sonnentag2003recovery},
  \item podejmowania nadobowiązkowych zadań \cite{salanova2005linking},
  \item proaktywne zachowania \cite{salanova2003perceived}.
\end{itemize}
\paragraph{Zaangażowanie grupowe}
Z interesujących badań, wykazano, że zaangażowanie nie jest cechą indywidualną, ale może być także przypisane do grup pracowników lub działów organizacji \cite{salanova2005linking,bakker2003multigroup}. Bakker i Schaufeli w 2001 wykazali, że zaangażowanie grup jest skorelowane z zaangażowaniem członków tych grup \cite{bakker2001burnout}. Pytanie brzmi, czy grupa wpływa na zaangażowanie członków, czy na odwrót? Słusznym wydaje się hipoteza, że jest to relacja
obustronna. Na pewno zaangażowanie członków grupy wiąże się ze zwiększoną dostępną pulą zasobów w pracy (np.: wsparcie społeczne współpracowników). Natomiast zasoby w pracy są tworzone przez poszczególnych pracowników. Ponadto wykazano, że zaangażowanie jest zaraźliwe, przechodzi między osobami będącymi blisko siebie \cite{bakker2003crossover}.

